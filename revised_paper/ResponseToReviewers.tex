\documentclass[11pt]{article}
\usepackage{geometry}                % See geometry.pdf to learn the layout options. There are lots.
\geometry{a4paper,
  top=25mm,
  bottom=30mm,
  left=25mm,
  right=25mm}                   % ... or a4paper or a5paper or ... 
%\geometry{landscape}                % Activate for for rotated page geometry
%\usepackage[parfill]{parskip}    % Activate to begin paragraphs with an empty line rather than an indent
\usepackage{graphicx}
\usepackage{amssymb}
\usepackage{epstopdf}
\usepackage{color}
\DeclareGraphicsRule{.tif}{png}{.png}{`convert #1 `dirname #1`/`basename #1 .tif`.png}

\title{Brief Article}
\author{The Author}
%\date{}                                           % Activate to display a given date or no date

\begin{document}
%\maketitle

\section*{Response to Reviewers}
\noindent We would like to convey our thanks to both reviewers for the constructive and positive reviews of our paper. 
In the following pages we have addressed each of the reviewers' comments -- with their comments in blue and our response in plain text.

\subsection*{Response to Reviewer 1}

{\color{blue} This study uses an ensemble of 50 simulations from an eddy-permitting ocean model to investigate the relative importance of forced (coherent) and intrinsic (chaotic) eddy responses. The simulations use the same wind forcing but differ from each other by adding a stochastic perturbation over 1960. The authors define the forced EKE as the ensemble mean EKE from the 50 simulations, and the chaotic EKE as the difference between each simulation and the ensemble mean. The authors find that the Southern Ocean eddy field is dominated by intrinsic processes instead of forced processes. In addition, the authors identified two timescales on which the wind stress and the forced eddy field are linked. 

Overall, the question is important, the approach is creative, the findings are significant, and the presentation is high quality. I think the paper is certainly publishable in JGR. I only have a few minor comments and questions, which I hope the authors can clarify before my recommendation of acceptance.\\}

\noindent Our thanks to the reviewer for their positive assessment. 
We have responded in detail to each of the reviewer's specific comments below.

{\color{blue} 
\begin{description}
\item[ Line 84: ] ``the Southern Ocean", missing words like EKE or eddy field?\\
{\color{black}  We agree and have changed to read ``Southern Ocean EKE''.}

\item[ Lines 140-141:] ``the member's global mean sea level anomaly", is this a single value or a time series? If this is a value, it's more likely to shift each ensemble member's sea surface height with a different value. But it wouldn't affect the sea level anomaly time series at each grid point. \\
{\color{black} This statement refers to a timeseries of the globally averaged sea level anomaly. We have clarified this statement to read:\\
 ``For each ensemble member, and at every time step, the member's global mean sea level anomaly is subtracted at every grid point  \ldots" }

\item[Lines 158-162:] The authors mention that the seasonal EKE variation dominates the statistics. To look at the forced response or the correlation between ensemble-mean EKE and wind stress, they deseasonalize the EKE time series. There is some missing information about how they smooth wind stress, the same as the deseasonalized EKE? Also, maybe some general information about the wind stress can be provided in the method section.\\
{\color{black} Yes, the seasonal cycle is also removed from the wind stress. We have added, on line 193 of the revised manuscript, the caveat that wind stress ``\ldots has been deseasonalised similarly to $E_i$ \ldots". }

\item[Lines 176-179:] ``Analogously, the intrinsic variance emerges from the variance, indicating the spread of each member from the ensemble mean"; I feel this sentence is not accurate. Normally the closest meaning of spread is the standard deviation. I would suggest the author rephrase it like this: ``For each time step, the intrinsic variance emerges from the variance of EKE difference between ensemble mean and each member of the 50 simulations.'' \\
{\color{black} Amended as suggested. }

\item[Lines 185-186:] between the EKE (...) AND WHAT? Missing words?\\
{\color{black} We have added ``and its forcing''. }

\item[Line 201:] show $\to$ shows\\
{\color{black} Amended. }

\item[Figure 1:] The ensemble mean EKE pattern is consistent with the mean EKE pattern from satellite altimetry. Since this study focuses more on the EKE variabilities, it would be helpful showing to what extent the averaged EKE anomaly time series (group members and ensemble mean) agree with the EKE anomaly time series from satellite altimetry. Also, please label the longitude, latitude, and the major topographic features. \\
{\color{black}  We have added latitude and longitude to Figure 1 as suggested (and made minor improvements to the plot).
We tested labelling major topographic features, but there are many features that could be labelled and it quickly gets messy, so we opted to leave the plot without labelling features.

Including the observed EKE trend is a good idea, and we have added this data to Figure~2(a) (green line), using EKE data calculated from Mart\`inez-Moreno et al. (2022).
This new plot shows that the observed EKE is indistinguishable from the individual ensemble members, but differs from the ensemble mean.}

\item[Line 231:] add citations (e.g., Meredith \& Hogg, 2006; Hogg et al., 2015) \\
{\color{black} Added as suggested }

\item[Line 231:] As found in this paper, the significance of the wind-leading lag correlation is affected by the size and location of the research area. To compare with previous studies, such as Meredith \& Hogg (2006), Hogg et al., (2015), doesn't it make more sense to use the same average boxes? \\
{\color{black} This line actually refers to the circumpolar average, and so provides a direct comparison, as the reviewer suggests.  
For the later parts of the paper where we examine sub-domains, we considered replicating the original Meredith \& Hogg (2006) boxes, but we found that the ensemble approach allows us to refine those boxes to be considerably smaller, while still obtaining meaningful results.
We could happily add an extra figure with the larger Meredith \& Hogg (2006) boxes, but consider this additional figure to be of marginal value to the reader.}

\item[Figure 2:] Based on the definition of EKE and wind stress, they should always be positive. To be accurate, I would suggest the authors label them clearly. Same comment for time series plots in figure 3 and figure 4. \\
{\color{black} We agree more information is needed here and have noted in the subtitle of figure 2(a) and the caption of figure 2 that the timeseries shows ``ensemble mean {\color{red}EKE anomaly} (red) along with wind stress {\color{red}anomaly} \ldots'', where red text has been added. 
The captions for figures 3 and 4 have been similarly clarified.}

\item[Line 239-240:] What's the T value for the 4 months and 30 months peaks? \\
{\color{black} The T-values in this case are 5.9 and 3.5 respectively. 
These values have been added to the manuscript as suggested.  }

\item[Line 241:] dominanent $\to$ dominant?\\
{\color{black} Corrected, thank you. }

\item[Line 274:] Southeast $\to$ Southwest\\
{\color{black} Amended. }

\item[Figure 3:] (a) A colorbar could be useful. Also, when you define the 3 boxes, do you use rectangular boxes like (70$^\circ$E-120$^\circ$E, 45$^\circ$S-60$^\circ$S)? The boxes shown in Figure 3(a) are more like irregular polygons. It will be better if the author can describe how those boxes are defined. \\
{\color{black}  We have added a colorbar as suggested. Regarding the irregular polygons -- yes, the boxes are long lines of latitude, but we were a little lazy in plotting these on the map projection. These boxes have now been fixed to follow latitude lines. }

\item[Line 276:] Do the envelopes in Figures 2(b), 3(d), and 4(d) indicate significant lag correlations? If so, it's not accurate to state "no significant response" here.\\
{\color{black} We agree with the reviewer here and have removed that statement. }

\item[Figure 4:] (a) Same comments as Figure 3 (a). In figure 4(b), the caption of the right axis is missing. \\
{\color{black}  Figure 4 now includes a colorbar and improved box boundaries as per Figure 3.
The missing axis label has been reinstated.}

\item[Figure 5:] Why is there a small missing longitudinal sector?\\
{\color{black} The small missing sector was a consequence of a particular choice of option when coarsening onto the larger grid, meaning that periodic fields were poorly mapped onto the circumpolar projection.  
We have now used a better option for coarsening and the field now covers the full 360$^\circ$ of longitudes.

We have also fixed the irregular polygon shapes of the boxes on Figure 5, as per Figures 3 and 4.}

\end{description}
}

\newpage

\subsection*{Response to Reviewer 2}

{\color{blue} I would like to begin by apologising to the authors for the delay in my review. Unfortunately, it became due at a time where there were multiple demands upon my time and I did a poor job at managing them. I apologise for the delay to the publication process this has caused.

The paper is an interesting and well written look at how variability in the Southern Ocean. It is presented very clearly, despite the complex nature of mesoscale eddy dynamics and the challenges of analysing what I am sure is a very large dataset. The analysis uses some simple statistical metrics to break the variability into a fraction that is due to intrinsic variability and that due to direct forcing from wind stress. The conclusions are well supported by the data and the figures are extremely clear. It is very easy for me to recommend publication as-is.

I have noted a few very minor things below, which the authors may wish to address at the proof stage.\\}

\noindent Our thanks to the reviewer for their suggestions; see the following for our responses.

{\color{blue} 
\begin{description}
\item[line 66;] dynamics, plural or singular? I'm not sure.\\
{\color{black} Singular -- amended to ``dynamic''. }

\item[line 128;] is this spinup forced by DFS5.2 as well?\\
{\color{black} Yes -- we have added that information to the revised manuscript. }

\item[line 139;] daily or day?\\
{\color{black} Altered to ``day''.}

\item[line 158;] is the subsampling achieved by selecting a single mean or by averaging them into monthly equivalents?\\
{\color{black} We don't quite understand the point that the reviewer is getting at here, but have clarified by adding the phrase ``\ldots   (by taking the weighted average over each month)". }

\item[line 161;] removing THE mean seasonal cycle\\
{\color{black} Amended as suggested.}

\item[line 197;] is the $r$ in Eq. (10) the correlation coefficient at the current lag, rather than at a specific lag?\\
{\color{black} Yes, $r$ is the coefficient at a given lag, as defined 9 lines earlier. }

\item[line 214;] ensemble member AS thin grey lines\\
{\color{black} Amended as suggested. }

\item[line 241;] dominanent to dominant?\\
{\color{black} Corrected. }

\item[line 372;] a reference would be helpful for anyone unfamiliar with the literature.\\
{\color{black} Reference to Arbic \& Flierl (2004) added. }

\end{description}
}

\end{document}  