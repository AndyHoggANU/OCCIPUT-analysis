\documentclass{agujournal2019}

\graphicspath{ {figures/} }
\usepackage{lineno,layouts,microtype}

\usepackage{natbib}

\linenumbers

\newcommand{\navidcomment}[1]{{\color{red} [#1]}}
\newcommand{\andycomment}[1]{{\color{Green} [#1]}}

%\usepackage[T1]{fontenc}
%\renewcommand*\familydefault{\sfdefault} 
%\usepackage{sansmath} 
%\sansmath

\newcommand{\mathbfit}[1]{\textbf{\textit{\textsf{#1}}}}
\usepackage{amsfonts, amssymb, amsmath}
\usepackage[labelfont=bf]{caption}
% \usepackage{doi}

\usepackage{comment}

%%%%%%%
% As of 2018 we recommend use of the TrackChanges package to mark revisions.
% The trackchanges package adds five new LaTeX commands:
%
%  \note[editor]{The note}
%  \annote[editor]{Text to annotate}{The note}
%  \add[editor]{Text to add}
%  \remove[editor]{Text to remove}
%  \change[editor]{Text to remove}{Text to add}
%
% complete documentation is here: http://trackchanges.sourceforge.net/
%%%%%%%


%% Enter journal name below.
%% Choose from this list of Journals:
%
% JGR: Atmospheres
% JGR: Biogeosciences
% JGR: Earth Surface
% JGR: Oceans
% JGR: Planets
% JGR: Solid Earth
% JGR: Space Physics
% Global Biogeochemical Cycles
% Geophysical Research Letters
% Paleoceanography and Paleoclimatology
% Radio Science
% Reviews of Geophysics
% Tectonics
% Space Weather
% Water Resources Research
% Geochemistry, Geophysics, Geosystems
% Journal of Advances in Modeling Earth Systems (JAMES)
% Earth's Future
% Earth and Space Science
% Geohealth
%
% ie, \journalname{Water Resources Research}

\journalname{JGR: Oceans}




\begin{document}
\justify
\title{Circumpolar variations in Southern Ocean eddy dynamics:\\An ensemble approach}

\authors{
Andrew McC. Hogg\affil{1}, Thierry Penduff\affil{2}, Sally E. Close\affil{3}, William K. Dewar\affil{4},\\Navid C. Constantinou\affil{1}\quad and\quad Josu{\'e} Mart{\'i}nez Moreno\affil{1}
}

\affiliation{1}{Research School of Earth Sciences and ARC Centre of Excellence for Climate Extremes,\\ Australian National University, Australia}
\affiliation{2}{Universit{\'e} Grenoble Alpes, CNRS, IRD, Grenoble-INP, IGE, Grenoble, France}
\affiliation{3}{Institut des Géosciences de l'Environnement, CNRS/Univ. Grenoble Alpes/G-INP/IRD, France}
\affiliation{3}{Florida State University, Tallahassee, FL, USA}

\correspondingauthor{Andrew McC. Hogg}{Andy.Hogg@anu.edu.au}

\begin{keypoints}
\item Variations in the Southern Ocean eddy field are dominated by intrinsic, rather than forced, processes.
\item The forced component of the the variance is governed by the local wind stress input.
\item The eddy field lags variations in wind stress, with two clear timescales emerging: one at 6-9 months, which we attribute to baroclinic instability, and one at 2-3 years, which we attribute to the delayed effect of topographic steering.
\end{keypoints}

\begin{abstract}
Circulation in the Southern Ocean is unique in global ocean.
The strong wind stress forcing and buoyancy fluxes, in concert with the lack of continental boundaries, conspire to drive the Antarctic Circumpolar Current replete with an intense eddy field.
The effect of Southern Ocean eddies on the ocean circulation is significant -- they modulate the momentum balance of the zonal flow, and the meridional transport of tracers and mass.
The strength of the eddy field is controlled by a combination of forcing (primarily thought to be wind stress) and intrinsic variability associated with the turbulent flow field itself.
Here, we present results from an eddy-permitting ensemble of ocean model simulations to investigate the relative contribution of forcing and intrinsic processes in governing the variability of the Southern Ocean eddy field.
We find that intrinsic processes dominate the eddy field.
Where correlations between the forcing and the eddy field exist, these interactions are dominated by two distinct timescale -- a fast baroclinic instability response; and a multi-year process owing to feedback between bathymetry and the mean flow.
These results suggest that understanding Southern Ocean eddy dynamics requires an ensemble approach to eliminate intrinsic variability, and therefore may not yield robust conclusions from observations alone.

\noindent \textbf{Plain language summary} \\
\noindent The Southern Ocean is the most turbulent region of the world's oceans.
The variations in this turbulence, which is often referred to as \emph{eddies}, is critical to understanding the evolution of the Southern Ocean under climate change.
But it's hard to get information about these eddies, because they occur on small scales in a large ocean basin that is poorly observed.
In addition, the observational record is quite short, which makes it more difficult to use these observations to study what controls variation in the eddy field.
For this reason, we take an eddy-permitting ocean model, and run it 50 times with the same forcing (but a slightly different initial state).
The chaotic nature of the turbulent ocean means that these models runs diverge into different states.
We thus use these simulations to study which eddy processes are intrinsic (that is, a consequence of the chaotic nature of turbulence) and which are forced by the external forcing that is common to all experiments (such as wind forcing).
We conclude that the Southern Ocean eddy field is dominated by intrinsic chaotic processes; but that the forced variability responds to wind on particular timescales that are controlled by the mechanisms that generate ocean turbulence. 
\end{abstract}




%% ------------------------------------------------------------------------ %%
%
%  TEXT
%
%% ------------------------------------------------------------------------ %%

\section{Introduction}

The Southern Ocean is unique in the global ocean; it is the one region without continents on its zonal boundaries, giving rise to the Antarctic Circumpolar Current (ACC) which flows eastward around the globe.
The ACC acts to connect the other major basins and thereby regulates climate and nutrients.
The Southern Ocean region is also a place where mid- and high-latitude ventilation of the oceans occurs, leading to carbon and heat uptake and controlling deep ocean stratification.
However, the Southern Ocean is also poorly observed (compared with other ocean basins) and it's unique properties mean that understanding the dynamics (with the aim of predicting future responses to climate change) need to be urgently constrained.

Another unique feature of the Southern Ocean is that it has a strong eddy field (Fu?).
This strong eddy field has suggested to be important in the Southern Ocean's response to change.
For example, the eddy saturation hypothesis (Hallberg, Hogg, Munday, constantinuo) suggests that the role of eddies in facilitating vertical momentum transport acts to limit the response of ACC transport to changing winds.
A similar dynamics, known as eddy compensation, describes the role of eddies in moderating the effect of wind-driven change on the Southern Ocean overturning circulation (Morrison).
Therefore, the response of fine scale transient motion in the Southern Ocean is likely important to characterising the dynamics of this region.

The strength of the Southern Ocean eddy field is usually characterised by extracting the kinetic energy of transient motions; referred to by oceanographers as eddy kinetic energy (EKE). 
It is important to highlight that EKE includes all transient motion, not just coherent vortices (Martinez Moreno 2019), and thus care needs to be taken in interpreting this metric.
Eddy kinetic energy has a complex relationship with the forces that drive the ocean circulation.
Meredith \& Hogg (2006) found significant variations in the area-averaged eddy field in some regions, and argued for a 2-3 year lag between events in wind stress forcing and EKE anomalies.
Patara et al. 2016, using a realistic high-resolution ocean model found that EKE does have a lagged response to wind stress anomalies, but that this relationship is variable around the Southern Ocean.
Idealised models over a wide range of parameter space (Sinha \& Abernathey) have highlighted that the timescale of the perturbation is critical in determining the EKE response, with shorter perturbations having a faster, Ekman-related response.
Thus, current knowledge suggests that there is a relationship between wind forcing and the Southern Ocean, but that the nature of this relationship, and the role of intrinsic variability, need to be clarified.

On longer timescales it has also been proposed that EKE has increased over recent decades (Hogg et al. 2015, Martinez Moreno 2019),
The robustness of this wind-EKE relationship in the Southern Ocean was recently investigated by Zhang et al. (2021).
These authors used crossover data from satellite observations (as in Hogg et al. 2015) to better estimate the EKE on regional scales. 
When fine graining these calculations it was found that only a couple of regions expressed a significant relationship between forcing and EKE.
This suggests that previous characterisations of the response may have been dominated by a small number of regional events, and that at other times the intrinsic variance may dominate the signal.

Thus, despite the proposed importance of the Southern Ocean eddy field to ocean circulation, the nature of the EKE variability remains unclear.
A primary complication is what can be inferred from an admittedly short satellite record; in particular, whether individual events can be attributed to forcing changes, or to intrinsic variability, or a combination of the two.
In this paper, we address this question by examining the intrinsic variability of the Southern Ocean eddy field in an ensemble of eddy-permitting  ocean models.
We use the OCCIPUT ensemble (Penduff et al .), a 50-member ensemble of hindcast simulations.
We examine both the intrinsic variance of the eddy field, and extract the ``forced'' (ensemble mean) component of the variability.
This variability is examined on a circumpolar and regional basis, to better understand the regional differences and processes which contribute to the eddy field.



\section{Methods}

\subsection{The OCCIPUT ensemble}

The methodology employed in this study is derived from the probabilistic approach to ocean modelling outinlined by \citet{Bessieres2017}.
We use output from the OCCIPUT (oceanic chaos – impacts, structure, predictability) ensemble of 50 eddy-permitting ocean model simulations, run in lock-step \citep{Leroux2018}.
This ensemble is based on the DRAKKAR-ORCA025 \citep[e.g.][]{Barnier2006} implementation of the NEMO modelling system \citep{Madec2012}.
The ensemble incorporates ocean-sea ice hindcast simulations for the period 1960-2015.
The ensemble uses the DRAKKAR Forcing Set DFS5.2 (Dussin 2016) with  0.25$^\circ$ lateral resolution and 75 vertical levels.


\subsection{Calculating geouv from SSH, removing trends, etc. (Sally)}

\subsection{TKE and filtering  to eliminate seasonal signal}

\subsection{Ratio of intrinsic to forced variance}

Calculate the time-variance of the ensemble-mean EKE:
$$\sigma^2_{\langle E \rangle} = \frac{1}{T} \sum_{t=1}^T \left(\langle E_i \rangle -  \overline{\langle E_i \rangle}\right)^2$$

Take the difference from the ensemble mean for each member:
$$E_i' = E_i - \langle E_i \rangle$$
This allows us to accumulate the ensemble variance,
$$\epsilon^2(t) = \frac{1}{N} \sum_{n=i}^N E_i'(t)^2$$
indicating the spread of each member from the ensemble mean.

Now calculate the ratio of intrinsic variance
$$ R_i =  \frac{\overline{\epsilon^2}}{\overline{\epsilon^2} + \sigma^2_{\langle E \rangle}}$$



\section{Results}

The intensity of the Southern Ocean eddy field is not uniform.
Snapshots of eddy kinetic energy (e.g. Fig.~\ref{Fig:1}a) show the occurrence of strong eddies which occur in the lee of subsurface topography and at the outlet from western boundary currents such as the Agulhas retroflection and Malvinas current.
The same patterns are evident in the ensemble mean of EKE (Fig.~\ref{Fig:1}b), although the signal of individual eddies is not longer apparent.
The strongest band of EKE approximately follows the path of the Antarctic Circumpolar Current, and EKE is generally weak south of 60$^\circ$S.
The patterns of EKE in this model broadly match the regional variations of EKE observed from satellite altimetry (REF), albeit at slightly lower intensity, as expected in an eddy-permitting model (e.g. Kiss et al. REF).

\begin{figure}[ht]
\begin{center}
\includegraphics[width=\hsize]{Figure1}
\caption{(a) EKE snapshot and (b) ensemble mean EKE  }
\label{Fig:1}
\end{center}
\end{figure}

For each of the 50 members of the OCCIPUT ensemble we take the EKE averaged over the entire Southern Ocean, and plot the deseasonalised EKE anomalies in thin grey lines in Fig.~\ref{Fig:2}(a). 
This plot highlights the considerable spread in EKE, even when averaged over the full circumpolar belt; in other words, there is a significant component of intrinsic variability in the Southern Ocean eddy field.
Nonetheless, when averaged over all ensemble members (red line in Fig.~\ref{Fig:2}a) the existence of a coherent (forced) component of eddy variability is revealed.
Averaged over this region the fraction, $R_i$, of intrinsic variance is 0.82, confirming the visual impression that intrinsic processes are dominant in the Southern Ocean eddy field.

\begin{figure}[t]
\begin{center}
\includegraphics[width=\hsize]{Figure2}
\caption{Eddy kinetic energy statistics over the Southern Ocean (40$^\circ$S--60$^\circ$S). (a) Spatially averaged EKE anomaly (relative to climatology) for individual ensemble members (grey) and the ensemble mean (red) along with wind stress (black); and (b) Time-lagged correlation for the ensemble mean (red) and individual ensemble  members (grey) -- with two individual ensemble members highlighted in magenta/orange.}
\label{Fig:2}
\end{center}
\end{figure}

Although Southern Ocean EKE is strongly intrinsic, there remains a significant component of forced variability.
Previous studies have suggested that there is a strong contribution of wind stress forcing upon EKE, and we therefore compare the forced variability with the variations in wind stress averaged over the same circumpolar belt (black line in Fig.~\ref{Fig:2}a).
This comparison suggests a relationship in which wind stress leads variations in EKE, consistent with previously published results.
However, the time-lagged correlations between wind stress and EKE suggests that this relationship is complex. 
The intrinsically variable nature of Southern Ocean eddies means that for some ensemble members, there is no meaningful correlation between wind stress and eddies (grey lines in Fig~\ref{Fig:2}b).
On the other hand, ensemble member 5 (magenta line) has a clear (and significant; T-value = 5.5) correlation with a 4-month lag, while member 25 (orange line) is correlated weakly at 6 months, and significantly correlated (T-value = 4.2) at a 24-month lag.
These isolated examples highlight how each ensemble member behaves differently, but the ensemble mean (red line) includes a significant correlation at $\sim$4 months and a second sifgnificant peak at $\sim$30 months.
Thus, there appear to be two distinct timescales of response of the Southern Ocean eddy field to wind stress. 

\begin{figure}[t]
\begin{center}
\includegraphics[width=\hsize]{Figure3}
\caption{Eddy kinetic energy statistics within sub-regions of the Southern Ocean. (a) Map showing Ensemble Mean EKE, along with 3 boxes over which a regional EKE analysis is applied; (b) Regional analysis of the Southeast Indian Ocean showing ensemble mean EKE in cyan, individual ensemble members EKE in grey and local wind stress averaged over the region in black; (c) Lagged correlation of ensemble mean EKE in each of the three regions; (d) Regional analysis of the Southwest Pacific Ocean showing ensemble mean EKE in red, individual ensemble members EKE in grey and local wind stress averaged over the region in black; and (3) Regional analysis of the South Atlantic Ocean showing ensemble mean EKE in orange, individual ensemble members EKE in grey and local wind stress averaged over the region in black. The fraction of intrinsic variance in each region is shown in the caption of panels (b), (d) and (e).}
\label{Fig:3}
\end{center}
\end{figure}

The ensemble of simulations shown here allow us to look in more detail at smaller regions of the Southern Ocean.
Calculating the variability of EKE in a smaller region has the advantage of isolating different processes which may occur in differing regions (for example, stronger topographic steering in places with steep bathymetry).
However, this advantage is partly offset by higher intrinsic variability in cases where the region of interest is so small that an individual eddy or event can have a large influence over the EKE timeseries.
In balancing these competing issues, we look at the variability within regions that span 15-20$^\circ$ in latitude and 30-40$^\circ$ in longitude, as shown in Figure \ref{Fig:3}(a).
We analyse the EKE timeseries averaged over these boxes -- including individual member EKE, ensemble mean EKE and lagged correlations between local wind stress forcing and the ensemble mean EKE in Fig. \ref{Fig:3}(b-e).

We first examine a region in the lee of Kerguelen Plateau in the Southeast Indian Ocean (cyan box in Fig.~\ref{Fig:3}a).
This region is characterised by high-frequency ($\sim$1 year) variations in EKE, with a relatively large forced component (the intrinsic variance fraction $R_i = 0.51$ which is smaller than the Southern Ocean average; Fig.~\ref{Fig:3}b).
The forced variation is clearly evident in the timeseries of from individual ensemble members; and this forced component is closely related to wind stress.
The lag between wind stress variations and ensemble mean EKE is short ($\sim$ 6 months; Fig.~\ref{Fig:3}c) with a single peak in the lagged correlation.
This region highlights a regime in which the eddy field responds rapidly to variations in the local wind stress.

In the Southeast Pacific Ocean (red box in Fig.~\ref{Fig:3}a) the situation is clearly different.
Here, the intrinsic variance fraction is similar to the Southeast Indian Ocean ($R_i=0.56$) but the timescale of the variability is much longer and there is no significant response to wind stress variations which occur at sub-annual scales (Fig.~\ref{Fig:3}d).
The peak in the lag correlation occurs at approximately 14 months in this region, and again has only a single clear peak.
Thus, this region varies slowly and consistently to multi-year variations in wind stress.

In the South Atlantic Ocean (orange box in Fig.~\ref{Fig:3}a) the system is more dominated by intrinsic variance ($R_i=0.65$) and is poorly correlated with wind stress forcing (Fig.~\ref{Fig:3}c,e), reinforcing the circumpolar variability of the EKE response to wind.
Other regions (see Fig. \ref{Fig:4}) highlight different aspects of the response;  with almost no correlation with wind forcing over the Southwest Indian Ocean (the Agulhas meander region; Fig. \ref{Fig:4}b) or the Central South Pacific (Fig. \ref{Fig:4}d).
In both of these regions, intrinsic variability dominates the signal.
On the other hand, the Southeast Pacific (north of the main pathway of the ACC) shows a strong and coherent multi-year response to wind stress; albeit with a very weak EKE signal (one tenth the magnitude of the core of the ACC).
The circumpolar variation in both EKE response times and intrinsic variability suggests that the two-timescale response seen in Fig. \ref{Fig:2} may be created by different processes, which each dominate in different regions of the Southern Ocean.


\begin{figure}[t]
\begin{center}
\includegraphics[width=\hsize]{Figure4}
\caption{Eddy kinetic energy statistics within sub-regions of the Southern Ocean. (a) Map showing Ensemble Mean EKE, along with 3 boxes over which a regional EKE analysis is applied; (b) Regional analysis of the Southwest Indian Ocean showing ensemble mean EKE in cyan, individual ensemble members EKE in grey and local wind stress averaged over the region in black; (c) Lagged correlation of ensemble mean EKE in each of the three regions; (d) Regional analysis of the South CentralPacific Ocean showing ensemble mean EKE in red, individual ensemble members EKE in grey and local wind stress averaged over the region in black; and (3) Regional analysis of the Southeast Pacific Ocean showing ensemble mean EKE in orange, individual ensemble members EKE in grey and local wind stress averaged over the region in black. The fraction of intrinsic variance in each region is shown in the caption of panels (b), (d) and (e).}
\label{Fig:4}
\end{center}
\end{figure}



\begin{figure}[t]
\begin{center}
\includegraphics[width=\hsize]{Figure5}
\caption{ TBC}
\label{Fig:5}
\end{center}
\end{figure}

The correlations between local wind and the ensemble mean EKE shows that, where forced variability in Southern Ocean EKE occurs, it can be well-explained by variations in wind stress.
However, these correlations are based purely on local wind stress -- averaged over the same area as the EKE statistics.
The existence of multi-year lags between the wind and the EKE suggests that local winds may not be the only source of energy for eddy generation; in particular, it is possible that energy could be advected a considerable distance downstream during this lag period.
To investigate this question we now take each of the regions outlined in Fig. \ref{Fig:3} and look at the spatial correlation between wind stress and the local ensemble mean EKE (Fig. \ref{Fig:5}.
To make this calculation, wind stress is first coarsened to a 2$^\circ \times $2$^\circ$ grid, and wind stress in each of those coarsened grid cells correlated with EKE at different lags.
In the Southeast Indian Ocean, Fig.~\ref{Fig:3}(c) shows correlation maxima at 7 months and 30 months; the spatial variation of this correlation is shown in Fig. \ref{Fig:5}(a,b) respectively.
These figures highlight a key feature of Southern Ocean wind stress, which is that there are strong correlations between wind stress at a given latitude; nonetheless, the maximum correlation between wind stress and EKE occurs within the EKE-averaging region.
This correlation is lower in magnitude at 30 months (consistent with Fig.~\ref{Fig:3}c), but at both 7 and 30 month lags the correlation with wind upstream of the EKE-averaging region is not stronger than within the EKE-averaging region.
A similar result is found in the Southwest Pacific Ocean (Fig.~\ref{Fig:5}c); wind stress correlations are relatively uniform across the Pacific Ocean owing the autocorrelation of winds, but the correlations are less circumpolar than the Southeast Indian Ocean.
Importantly, there is no suggestion of a strong correlation with wind stress upstream of the EKE-averaging region.
In the South Atlantic, the EKE is not strongly correlated with wind stress, either in the local region or elsewhere in the Southern Ocean (Fig.~\ref{Fig:5}d);
Thus, these spatial maps demonstrate that, where strong forced variability in the EKE exists, it is most strongly linked to local wind stress, with no suggestion of upstream or remote wind input playing a strong role. 

\section{Discussion}
How do we interpret intrinsic variance ratio in this case?

Figure 6 - Ri ratio plot, and other global features


What can we infer about real world?

Propose hypotheses for two timescales (do we need to test these at all?)



\section{Conclusions}
\begin{itemize}
    \item Intrinsic variance is large.
    \item  Forced component exists and has two timescales.
    \item Short timescale is local eddy response - mixed  barotropic/baroclinic?
    \item Long timescale requires topographic feedback.

\end{itemize}



\acknowledgments
We acknowledge NCI for everything they do for us.

\bibliography{references}

\end{document}