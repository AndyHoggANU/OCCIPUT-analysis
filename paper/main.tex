
\documentclass{agujournal2019-navid}

\graphicspath{ {figures/} }
\usepackage{lineno,layouts,microtype}


% \linenumbers

\newcommand{\navidcomment}[1]{{\color{red} [#1]}}
\newcommand{\andycomment}[1]{{\color{Green} [#1]}}

%\usepackage[T1]{fontenc}
\renewcommand*\familydefault{\sfdefault} 
\usepackage{sansmath} 
\sansmath

\newcommand{\mathbfit}[1]{\textbf{\textit{\textsf{#1}}}}
\usepackage{amsfonts, amssymb, amsmath}
\usepackage[labelfont=bf]{caption}
% \usepackage{doi}

\usepackage{comment}

%%%%%%%
% As of 2018 we recommend use of the TrackChanges package to mark revisions.
% The trackchanges package adds five new LaTeX commands:
%
%  \note[editor]{The note}
%  \annote[editor]{Text to annotate}{The note}
%  \add[editor]{Text to add}
%  \remove[editor]{Text to remove}
%  \change[editor]{Text to remove}{Text to add}
%
% complete documentation is here: http://trackchanges.sourceforge.net/
%%%%%%%


%% Enter journal name below.
%% Choose from this list of Journals:
%
% JGR: Atmospheres
% JGR: Biogeosciences
% JGR: Earth Surface
% JGR: Oceans
% JGR: Planets
% JGR: Solid Earth
% JGR: Space Physics
% Global Biogeochemical Cycles
% Geophysical Research Letters
% Paleoceanography and Paleoclimatology
% Radio Science
% Reviews of Geophysics
% Tectonics
% Space Weather
% Water Resources Research
% Geochemistry, Geophysics, Geosystems
% Journal of Advances in Modeling Earth Systems (JAMES)
% Earth's Future
% Earth and Space Science
% Geohealth
%
% ie, \journalname{Water Resources Research}

% \journalname{Geophysical Research Letters}

\usepackage{scalerel, stackengine}
\setstackEOL{\#}
\stackMath
\def\hatgap{2pt}
\def\subdown{-2pt}
\newcommand\reallywidehat[2][]{ \renewcommand\stackalignment{l} \stackon[\hatgap]{#2}{ \stretchto{
    \scalerel*[\widthof{$#2$}]{\kern-.6pt\bigwedge\kern-.6pt}
    {\rule[-\textheight/2]{1ex}{\textheight}}}
    {0.5ex}_{\smash{ \belowbaseline[\subdown]{\scriptstyle#1} }}
}}


%% MACROS

% Vector calculus 
  \newcommand{\p}		{\partial}
  \newcommand{\bnabla}	{\boldsymbol \nabla}
  \newcommand{\grad}	{\bnabla}
\renewcommand{\div}	    {\bnabla \bcdot}
  \newcommand{\bcdot}	{\boldsymbol \cdot}
  \newcommand{\lap}		{\triangle}

% Boldsymbols
\newcommand{\bu}		{\boldsymbol{u}}
\newcommand{\bx}		{\mathbfit x}
\newcommand{\bU}		{\mathbfit{U}}
\newcommand{\bX}		{\mathbfit{X}}
\newcommand{\bxh}		{\hspace{0.1em} \boldsymbol{\hat x}}
\newcommand{\byh}		{\hspace{0.1em}\boldsymbol{\hat y}}
\newcommand{\bzh}		{\hspace{0.1em}\boldsymbol{\hat z}}
\newcommand{\bnh}		{\hspace{0.1em}\boldsymbol{\hat n}}
\newcommand{\bxi}		{\ensuremath {\boldsymbol {\xi}}}

% Greek
\newcommand{\ep}		{\epsilon}
\newcommand{\om}		{\omega}
\newcommand{\kap}		{\kappa}
\newcommand{\sig}		{\sigma}
\newcommand{\gam}	  {\gamma}
\newcommand{\lam}		{\lambda}

% Romans
\newcommand{\ee}		{\mathrm{e}}
\newcommand{\ii}		{\mathrm{i}}
\newcommand{\dd}		{{\rm d}}
\newcommand{\id}		{{\, \rm d}}

% Misc
\newcommand{\Dt}[1]	    {\mathrm{D}_t #1}
\newcommand{\half}		{\tfrac{1}{2}}
\newcommand{\where}	    {\qquad \text{where} \qquad}
\newcommand{\andand}	{\qquad \text{and} \qquad}
\newcommand{\com}		{\, ,}
\newcommand{\per}		{\, .}
\newcommand{\defn}	    {\ensuremath{\stackrel{\mathrm{def}}{=}}}
\renewcommand{\equiv} {\ensuremath{\stackrel{\mathrm{def}}{=}}}
\newcommand{\av}[1]	    {\left \langle {#1} \right \rangle}
\newcommand{\beq}		{\begin{equation}}
\newcommand{\eeq}		{\end{equation}}
\newcommand{\Pa}		{\mathrm{N}\,\mathrm{m}^{-2}}
\newcommand{\rhom} {\rho_{\mathrm{m}}}
\newcommand{\ws} {\mathrm{WS}}
\newcommand{\tfs} {\mathrm{TFS}}
\newcommand{\ifs} {\mathrm{IFS}}
\newcommand{\bd} {\mathrm{BD}}
\newcommand{\hb} {h_{\mathrm{bot}}}
\newcommand{\pb} {p_{\mathrm{bot}}}



\hypersetup{
	breaklinks,
	colorlinks=true,
	linkcolor=Blue,
  urlcolor=Purple,
	citecolor=Purple,
%	allcolors=black,
	pdfauthor={N. C. Constantinou and A. McC. Hogg}
 }


\begin{document}
\justify
\title{Circumpolar variations in Southern Ocean eddy dynamics:\\An ensemble approach}

\authors{
Andrew McC. Hogg\affil{1}, Thierry Penduff\affil{2}, Sally E. Close\affil{3}, William K. Dewar\affil{4},\\Navid C. Constantinou\affil{1}\quad and\quad Josu{\'e} Mart{\'i}nez Moreno\affil{1}
}

\affiliation{1}{Research School of Earth Sciences and ARC Centre of Excellence for Climate Extremes,\\ Australian National University, Australia}
\affiliation{2}{Universit{\'e} Grenoble Alpes, CNRS, IRD, Grenoble-INP, IGE, Grenoble, France}
\affiliation{3}{Institut des Géosciences de l'Environnement, CNRS/Univ. Grenoble Alpes/G-INP/IRD, France}
\affiliation{3}{Florida State University, Tallahassee, FL, USA}

\correspondingauthor{Navid C. Constantinou}{navid.constantinou@anu.edu.au}

% \begin{keypoints}
% \item An isopycnal layered model, with a varying number of fluid layers, is used to assess relative importance of barotropic and baroclinic processes in the Southern Ocean.
% \item Both baroclinic and barotropic flows exhibit regimes in which mean zonal transport is insensitive to wind stress.
% \item Eddies actively shape the time-mean flow, irrespective of the instabilities from which they originate. 
% \end{keypoints}




\begin{abstract}
We study here this and that...

\noindent \textbf{Plain language summary} \\
\noindent In plain words, we do this and that...
\end{abstract}

%% ------------------------------------------------------------------------ %%
%
%  TEXT
%
%% ------------------------------------------------------------------------ %%

\section{Introduction}

Southern Ocean eddies are important to the climate response - for compensation, saturation, etc.

Proxy for eddy strength is Transient KE (TKE), often called eddy KE. Define.

TKE responds to variations in wind stress forcing, but with a lag. Studied by many. This lag is relevant for eddy saturation and LFV.

Confusion about what this lag might be. It seems to be event- and region-dependent. Also, not clear whether it emerges from the noise in either model simulations or observations.

So, let’s reduce the noise with an ensemble approach. Look at both SO-wide and regional responses.


\section{Methods}

Model description (Thierry)

OCCIPUT strategy (Thierry)

Calculating geouv from SSH, removing trends, etc. (Sally)

Paragraph on TKE and filtering  to eliminate seasonal signal (Andy)

Paragraph on intrinsic-forced ratio and its limitations.


\section{Results}
Figure 1 - (a) TKE snapshot and (b) TKE climatology

Figure 2 - Entire Southern Ocean:(a,b) cherrypicked experiments TKE + lag; (c) ensemble mean TKE + lag

Figure 3 - 4 or 6 sectors TKE + lag

Figure 4 - interesting regions - again 4 or 6 of them, TKE + lags

Figure 5 - Ri ration plot, and other global features

\section{Discussion}
How do we interpret intrinsic variance ratio in this case?

What can we infer about real world?

Propose hypotheses for two timescales (do we need to test these at all?)



\section{Conclusions}
\begin{itemize}
    \item Intrinsic variance is large.
    \item  Forced component exists and has two timescales.
    \item Short timescale is local eddy response - mixed  barotropic/baroclinic?
    \item Long timescale requires topographic feedback.

\end{itemize}



\acknowledgments
We acknowledge NCI for everything they do for us.

\bibliography{basic_references}

\end{document}